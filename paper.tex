\documentclass[letterpaper]{article}
\usepackage[utf8]{inputenc}
\usepackage[english]{babel}
\usepackage{/home/adria/Dropbox/latex/aaai}
\usepackage{times}
\usepackage{helvet}
\usepackage{courier}
\usepackage{color}
\usepackage{natbib}
\usepackage{amssymb,amsmath,amsfonts,amsthm}
\usepackage{dsfont}
\usepackage{mathtools}
\usepackage{enumitem}   
\usepackage{graphicx}
\usepackage{hyperref}
\usepackage{import}
\input{/home/adria/Dropbox/latex/acronyms.tex}

\usepackage{algorithm}
\usepackage[end]{algpseudocode}
%\let\Algorithm\algorithm
%\renewcommand\algorithm[1][]{\Algorithm[#1]\setstretch{1.25}}

\usepackage{calc}
\def\svgscale{1}
\newcommand{\inkscapefig}[2]{
\begin{figure}[hbtp]
\begin{center}
  \subimport{img/}{#1.pdf_tex}
\end{center}
\caption{#2\label{fig:#1}}
\end{figure}
}

\usepackage[plain]{fancyref}
% Change references from capital/normal letters
\newcommand{\ffref}{\Fref}
% Algorithm and subsection references
\newcommand*{\fancyrefalglabelprefix}{alg}
\newcommand*{\fancyrefsubseclabelprefix}{subsec}
\frefformat{plain}{\fancyrefalglabelprefix}{%
algorithm\fancyrefdefaultspacing#1%
}
\Frefformat{plain}{\fancyrefalglabelprefix}{%
Algorithm\fancyrefdefaultspacing#1%
}
\frefformat{plain}{\fancyrefsubseclabelprefix}{%
subsection\fancyrefdefaultspacing#1%
}
\Frefformat{plain}{\fancyrefsubseclabelprefix}{%
Subsection\fancyrefdefaultspacing#1%
}
\frefformat{vario}{\fancyrefalglabelprefix}{%
algorithm\fancyrefdefaultspacing#1 on page~#2%
}
\Frefformat{vario}{\fancyrefalglabelprefix}{%
Algorithm\fancyrefdefaultspacing#1 on page~#2%
}
\frefformat{vario}{\fancyrefsubseclabelprefix}{%
subsection\fancyrefdefaultspacing#1 on page~#2%
}
\Frefformat{vario}{\fancyrefsubseclabelprefix}{%
Subsection\fancyrefdefaultspacing#1 on page~#2%
}

\bibliographystyle{/home/adria/Dropbox/latex/aaai}

\newcommand{\vbar}{\,|\,}
\newcommand{\mvbar}{\,\middle|\,}
\let\Pr\undefined
\DeclareMathOperator*{\Pr}{\mathbb{P}}
\DeclareMathOperator*{\E}{\Ex}
\newcommand{\Prob}[3]{\Pr_{#1 \sim #2 } \left[ #3 \right]}
\newcommand{\Expect}[3]{\E_{#1 \sim #2 } \left[ #3 \right]}
\newcommand{\Ex}{\mathbb{E}}
\newcommand{\tp}{\top}
\newcommand{\vx}{{\bf x}}
\newcommand{\vv}{{\bf v}}
\newcommand{\vX}{{\bf X}}
\newcommand{\va}{{\bf a}}
\newcommand{\vA}{{\bf A}}
\newcommand{\vb}{{\bf b}}
\newcommand{\vB}{{\bf B}}
\newcommand{\vc}{{\bf c}}
\newcommand{\vd}{{\bf d}}
\newcommand{\vC}{{\bf C}}
\newcommand{\vD}{{\bf D}}
\newcommand{\mis}{\text{mis}}
\newcommand{\obs}{\text{obs}}
\newcommand{\misx}{{\text{mis}_\vx}}
\newcommand{\obsx}{{\text{obs}_\vx}}
\newcommand{\misv}{{\text{mis}_\vv}}
\newcommand{\obsv}{{\text{obs}_\vv}}
\newcommand{\Indicator}[1]{\mathds{1}(#1)}

\setlength{\pdfpagewidth}{8.5in}
\setlength{\pdfpageheight}{11in}
\pdfinfo{
/Title (Missing data imputation with iterated Gaussian Processes)
/Author (Adrià Garriga-Alonso, Jinsung Yoon, Mihaela van der Schaar)}
\setcounter{secnumdepth}{0}

\title{Missing data imputation with iterated Gaussian Processes}
\author{Adrià Garriga-Alonso, Jinsung Yoon, Mihaela van der Schaar}
\begin{document}
\nocopyright
\maketitle

\begin{abstract}
\begin{quote}
  To be written.
\end{quote}
\end{abstract}

\section{Introduction}


\ac{MICE} \citep{van1999flexible} and MissForest
\citep{stekhoven2011missforest} use normal classification and regression models
to perform data imputation. Inspired by their success, we try to extend their approach to
a Bayesian framework, in which uncertainty about the missing data is explicitly
propagated.

The main contributions of this paper are:
\begin{itemize}
\item An iterative Bayesian procedure to estimate the probability distribution
of the missing measurements given the observed measurements,
$p(x_{\cdot,\text{mis}}\vbar x_{\cdot,\text{obs}})$. The same procedure can be used for direct
inference (without the intermediate step of imputation).

\item An inference algorithm for the Variational Infinite GMM
 \citep{blei2006variational,BishopChristopherM2006Pram} that extends to
partially observable inputs \citep{DIZIO20075305}.

\item A generalization of the \acs{RBF} kernel to uncertain inputs with \ac{MoG}
densities.

\end{itemize}
\section{Problem setup}
Let $\bf X$ be a $n \times d$ data matrix, which has some missing values.
We denote the $i$th data point with ${\bf X}_{i,:}$ or $\vx_i$ and the $j$th
dimension of all points by ${\bf X}_{:,j}$. $\vx_{i,\text{mis}}$ are the missing
values of the $i$th point, and $\vX_{\text{obs}_j}$ are the points where the
$j$th dimension is observed.

We assume that
these points are drawn independently from the same distribution, $\vx_i \sim
\mathcal{D}$. We would like to estimate the probability distribution of the
missing values given the observed values, $p(\vx_{i,\mis} \vbar
\vx_{i,\obs})$, for all points $i$.

In an ideal Bayesian setting, we would proceed as
follows. First, we make assumptions about what the distribution $\mathcal{D}$ can be: we
choose an expression $p(\vX \vbar \theta)$ for the likelihood of the data given
some parameters $\theta \in \Theta$, and a prior distribution $p(\theta)$ for these
parameters.
%parametrised by $\theta$ for the likelihood of the
%missing
%values given the observed values $p(\vX_{:,\mis}, \vX_{:,\obs} \vbar \theta)$;
%and a prior distribution for the parameters, $p(\theta)$.
Then, we calculate the posterior distribution of the parameters given the data
that is available to us, $p(\theta \vbar \vX_{:,\obs}) \propto
p(\vX_{:,obs} \vbar \theta)p(\theta)$. Using that, we can calculate the desired
distribution, $p(\vX_{:,\mis} \vbar \vX_{:,\obs}) = \int_\Theta
p(\vX_{:,\mis}\vbar \theta)p(\theta \vbar \vX_{:,\obs}) d\theta$. 
This gives us a complete specification of all the information that can be
obtained from the observed data, given our assumptions
about the form of $p(\vX | \theta)$ and $p(\theta)$.

However, this inference is intractable for all but the simplest of models. In
practice, we approximate the posterior $p(\vX_{:,\mis} | \vX_{:, \obs})$ with
another distribution $q(\vX_{:,\mis})$, for example using variational inference.
Because of the \ac{iid} assumption, the distribution can be factored across data
points, $q(\vX_{:,\mis}) = \prod_{i=1}^n q(\vx_{i,\mis})$.


This paper describes two algorithms to estimate $q(x_i)$. The first one,
based on the variational infinite \acl{GMM}
\citep{blei2006variational,BishopChristopherM2006Pram}, follows the procedure
just described to produce an estimation of the data density with \acf{MoG} form:
\begin{equation}
  q(\vx_i) = \sum_{k=1}^K \pi_k \: \mathcal{N}(\vx_i \vbar \mu_k, \Sigma_k)
  \label{eq:point-factored-mog}
\end{equation}
Given this distribution, it is possible to analytically compute
the conditional distribution
$q(x_{i,\mis} \vbar x_{i,\obs}; \left\{\mu_k, \Sigma_k\right\}_k)$.

As MissForest \citep{stekhoven2011missforest} shows, a crucial component of
having an accurate estimate of the missing data is finding correlations between
variables. However, Gaussian
distributions can only reflect linear correlations and, intuitively, \aclp{MoG}
can only reflect softly piece-wise linear ones.

To address this shortcoming, we introduce a second algorithm, based on
MissForest and \acfp{GP}. The distribution
defined by this algorithm is factorized over data points and dimensions:

\begin{equation}
 q(\vx_i)= \prod_{j=1}^d \mathcal{N}(x_{i,j} \vbar \mu_{i,j}, \sigma_{i,j})
  \label{eq:point-factored-gaussian}
\end{equation}

Note that this distribution has a mean and variance for every data point and
dimension, so it is not strictly less expressive than the one in
\ffref{eq:point-factored-mog}.
The distribution also satisfies two basic goals of data imputation. One, to find
an estimate of the missing values, $\mu_{i,j}$, that is as close as possible to
the real values. Two, to quantify how much these estimations may differ from the
real value, with $\sigma_{i,j}$.

%Following the work of
%\citet{nebot2010kernel}, we take the
%distribution estimated 
%We describe one algorithm for estimating
The algorithm operates by iteratively refining each $\mu_i$ and $\sigma_i$ from
a previous estimate, much like MissForest. A summary of the algorithm can be
found in \ffref{fig:overview}.

\begin{figure}[hbtp]
\subimport{img/}{overview.pdf_tex}
\caption{Overview of the Iterative \acp{GP} algorithm\label{fig:overview}}
\end{figure}

\section{Infinite \ac{GMM} with missing data}

We estimate the probability density of the data using a variational
approximation to an infinite \ac{GMM}, which can be done even in the presence of
missing inputs\footnote{To write: show how.}. Once
inferred, the \ac{GMM} gives rise to a \ac{MoG} density for each point's missing
values:
\begin{equation}
  q^0(\vx_{i,\text{mis}} \vbar \vx_{i,\text{obs}}) = \sum_{k=1}^K w_k \mathcal{N}(\vx_{i,\text{mis}} \vbar \vx_{i,\text{obs}}, \mu_k, \Sigma_k)
  \label{eq:ini-mog}
\end{equation}
where $w_k$ is the weight of the $k$th Gaussian cluster. The superindex of the
distribution name $q$ indicates the iteration of the algorithm it pertains to.

\section{Iterative \aclp{GP}}

We briefly review \acp{GP} \citep{rasmussen2006gaussian}. A function $f(\vx)$ is
drawn from a \acl{GP}, $f(\vx) \sim \mathcal{GP}(0, k(\vx, \vx'))$, if $k(\vx,
\vx) = \mathcal{E}[f(\vx)f(\vx')]$. $k$ is called the \emph{kernel function},
and it completely specifies a \ac{GP} with the mean function, which here we have
defined to be zero. Intuitively, kernel function specifies how similar the
outputs $f(\vx)$ and $f(\vx')$ of two points are.

A commonly-used kernel function is the \ac{RBF} kernel
\citep[Equation~5.1]{rasmussen2006gaussian}:
\begin{equation}
k(\vx_a, \vx_b) =  \sigma_f^2 \exp\left( -\frac{1}{2}(\vx_a - \vx_b)^\tp M
    (\vx_a - \vx_b) \right) + \sigma_n^2 \delta_{ab}
\label{eq:rbf-definition}
\end{equation}
with $M$ usually a diagonal matrix, sometimes just a scalar; and $\delta_{ab}$
is a Kronecker delta.

The function is undefined when one or more of the values $x_i$ are missing.
However, if we have a probability distribution over the missing values, we can
compute the expectation of kernel, which is a valid kernel
\citep{nebot2010kernel}:
\begin{equation}
k(\vx_a, \vx_b) =
\E_{\begin{array}{c}
  \vx_a \sim q^0(\cdot \vbar \vx_{a,\text{obs}}) \\
  \vx_b \sim q^0(\cdot \vbar \vx_{b,\text{obs}}) \end{array}} \left[
\hat{k}(\vx_a, \vx_b) \right]
\label{eq:kernel-fun}
\end{equation}

Note that the distribution $q^0(\vx_a \vbar \vx_{a,\text{obs}})$ is a Kronecker
delta if $\vx_a$ is fully observed; otherwise it is a \ac{MoG} conditioned on
the observed values. See the Appendix for the calculation of \ref{eq:kernel-fun}.

This kernel function is used to infer a \ac{GP} for each dimension $j\in
1,\dots,d$. The $j$th \acp{GP} is trained to output the value of the $j$th
dimension of the input, given the probability distributions of all dimensions
except $j$. The \ac{GP} learns only from the points in
which the $j$th dimension is present, that is, ${\bf X}_{\text{obs}_j,:}$.
$\text{obs}_j$ denotes the indices of the points in which dimension $j$ is
observed.

The \acp{GP} for each dimension, when evaluated for each point in $\vX$ and
combined, form a factored Gaussian distribution with diagonal covariance. This
distribution, now named $q^1$, is an approximation to
\Fref{eq:point-factored-gaussian}, same as $q^0$.

The form of $q^1$ is a special case of a \ac{MoG}, so the same procedure can be
applied to produce $q^2, q^3,\dots,q^T$. Our algorithm repeats the procedure
until the distributions converge, that is, the norm of the difference between
the parameters of two consecutive distributions is less than $\varepsilon$.

The algorithm is summarized in \Fref{fig:overview} and \Fref{alg:main-summary}.
In the next sections, we explain how to compute \Fref{eq:ini-mog} and
\Fref{eq:kernel-fun}.

\begin{algorithm}[h]
\begin{algorithmic}
\State {\bf Require:} ${\bf X}$ an $n \times d$ matrix with some missing values.
\State $m_\textsc{igmm} \gets$ fit Infinite \ac{GMM} to $\vX$.
\For{each data point $i=1,\dots,n$}
\State $q^0(\vx_i) \gets (\delta_{x_{i,\text{obs}}}, \text{MoG}(\vx_{i,\text{mis}} \vbar \vx_{i,\text{obs}}))$ with $m_\textsc{igmm}$.
\EndFor
\State $t \gets 0$. The initial estimated density is $q^0(\vX_i)$.
\While{$q^t(\vX)$ hasn't yet converged}
  \For{each column $j=1,\dots,d$}
    \State Fit a \ac{GP}: $\vX_{\text{obs}_j, j} \sim q^t(\vX_{\text{obs}_j, -j})$
     using kernel function in \Fref{eq:kernel-fun}.
    \State $q^{t+1}(\vX) \gets $ \ac{GP}-predicted mean and variance of missing values
    $\vX_{:,\text{mis}}$.
    \State $n \gets n + 1$
  \EndFor
\EndWhile
\State \Return last predicted distribution, $q^T$.
\end{algorithmic}
\caption{Iterated \acl{GP}}
\label{alg:main-summary}
\end{algorithm}

\section{Experiments}

We try to recover artificially-induced-missing data, especially from MAR and
MNAR settings; where our method might outperform MissForest, if we're lucky. 

There will hopefully be tables and graphs.

\section{Related Work}
MissForest do imputation and multiple imputation \citep{stekhoven2011missforest}.

MICE do multiple imputation and it is basically the same as MissForest
\citep{van1999flexible}.

The Variational \ac{GPLVM} \citep{damianou2016variational} also perform
inference and prediction in the presence of missing data. However, they require
a subset of the data that is complete, in order to bootstrap their method; which
makes it biased for MAR data.


The expectation over the kernel was developed by \citet{nebot2010kernel} and
\citet{hernandez2015study}. They used a factored Gaussian model of the data, in
which every point has an associated Gaussian. They also assumed that the
dimensions were independent.

%Apparently \citet{zhu2011missing} estimate density with kernel functions. I don't
%understand the paper.

\citet{iwata2012warped} also estimate a density model, which is effectively
\ac{GPLVM} but with a latent Gaussian Mixture Model instead of a single
Gaussian.

\bibliography{/home/adria/Dropbox/papers/already-read.bib}

\section{Appendix}

\subsection{Multidimensional \ac{RBF} kernel with partially observed \ac{MoG}}
We want to calculate $k(\vx, \vv)$, where any of $x_1,\dots,x_d = \vx$ or
$v_1,\dots,v_d = \vv$ may be missing. Let $\vx$ and $\vv$ be random variables denoting
two inputs to the kernel. Let $\text{mis}(\vx)$ and $\text{obs}(\vx)$ be the
dimensions that are missing and observed in $\vx$, respectively. Let
$\vx_{\text{mis}(\vx)}$, shortened as $\vx_\text{mis}$, be the random variable
denoting the values of the variables missing in $\vx$, analogously with
$\vx_{\text{obs}(\vx)}$ and $\vx_\text{obs}$.

The probability density of $\vx_\text{mis}$ is a \acl{MoG}:
\begin{equation*}
  p(\vx_\text{mis} ) = \sum_{k=1}^K w^\vx_k \, \mathcal{N}(\vx_\text{mis} \vbar \mu_k^\vx, \Sigma_k^\vx)
\end{equation*}

with $K$ clusters, means $\mu_k^\vx$, covariances $\Sigma_k^\vx$ and importances
$w_k^\vx$. We write $p(\vx_\text{mis})$ instead of $p(\vx_\text{mis} \vbar
\vx_\text{obs})$ because $\vx_\text{mis}$ is conditionally independent from
$\vx_\text{obs}$ given the means, covariances and importances of all the
clusters.

The distributions of $\vx_\mis$ and $\vv_\mis$ come from the same \ac{GMM}.
However, $\vv_\mis$ has instead $L$ Gaussians in its input distribution, which may be
different from $K$. This is because, to reduce the amount of computation
required, we may prune Gaussians according to their responsibility for a point.

Note that
$\text{mis}_\vx$ and $\text{mis}_\vv$ are not the same set of dimensions in
general. Thus, the distributions $p(\vx_\text{mis})$ and $p(\vv_\text{mis})$ may
not have the same number of variables.

The probability density of $\vx_\text{obs}$ is a Dirac delta at the point where
the variables are actually observed, $\mu^\vx_{\text{obs}(\vx)}$, shortened as
$\mu^\vx_\text{obs}$. We denote the distribution as
$\delta(\vx_\text{obs}-\mu^\vx_\text{obs})$.

We are ready to define the probability distribution of a partially observed
point. The defined distributions for $\vx_\text{mis}$ and $\vx_\text{obs}$ are
conditionally independent. Thus:

\begin{equation}
p(\vx) = \left(\sum_{k=1}^K w_k^\vx \, \mathcal{N}(\vx_\text{mis}| \mu^\vx_k,
  \Sigma^\vx_k)\right) \delta(\vx_\text{obs}-\mu^\vx_\text{obs})
\label{eq:pdf-partially-observed-long}
\end{equation}

We make the Gaussian explicit and separate the terms that contain $\vx$ from the
ones that don't, inside the sum:
\begin{equation}
p(\vx) = \sum_{k=1}^K \tau_k^\vx g_k^\vx
\label{eq:pdf-partially-observed}
\end{equation}

With:
\begin{equation*}
\tau_k^\vx = \left(2\pi\right)^{-|\misx|/2} |\Sigma_k^\vx|^{-1/2}w_k^\vx
\end{equation*}
\begin{equation*}
  \begin{aligned}
g_k^\vx &= \exp\left[-\frac{1}{2}\left(\vx_\text{mis} -
    \mu_k^\vx\right)^\tp(\Sigma_k^\vx)^{-1}\left(\vx_\text{mis} - \mu_k^\vx\right)\right] \\
&\hspace{4em} \delta(\vx_\text{obs}-\mu^\vx_\text{obs})
  \end{aligned}
\end{equation*}

Following \fref{eq:kernel-fun} we calculate:
\begin{equation}
  k(\vx, \vv) = \E_\vx \E_\vv \left[ \hat{k}(\vx, \vv) \right]
  \label{eq:expect-kernel}
\end{equation}

Here we distinguish three cases.
\begin{enumerate}
\item If $\vx$ and $\vv$ are both fully observed, then straightforwardly $k(\vx,
\vv) = \hat{k}(\vx, \vv)$.

\item If $\vx$ and $\vv$ are the same observation, then they are not
  independent. We substitute $\vv$ for $\vx$ in \fref{eq:expect-kernel}:
\begin{equation}
  k(\vx, \vx) = \E_\vx \left[\hat{k}(\vx, \vx) \right] = \int_{-\infty}^\infty
k(\vx, \vx) p(\vx) d\vx
  \label{eq:same-inputs}
\end{equation}

\item If $\vx$ and $\vv$ are different observations, by
the \ac{iid} assumption they are independent: $p(\vx, \vv) = p(\vx)p(\vv)$. Thus
we can write \fref{eq:expect-kernel} as:
\begin{equation}
  k(\vx, \vv) = \E_\vx \E_\vv \left[ \hat{k}(\vx, \vv) \right] = 
  \int_{-\infty}^\infty k(\vx, \vv) p(\vx) p(\vv) d\vx d\vv
  \label{eq:independent-inputs}
\end{equation}
\end{enumerate}

The developments above were kernel-agnostic. However, from now on we are
concerned exclusively with the \ac{RBF} kernel, defined in
\fref{eq:rbf-definition}.

\subsubsection{Case 2: $\vx=\vv$, with \ac{RBF} kernel.}
We substitute into \fref{eq:same-inputs} the definition from
\fref{eq:rbf-definition}:

\begin{equation*}
\begin{aligned}
 k(\vx, \vx) &= \int \bigg[\sigma_f^2 \exp\left( -\frac{1}{2}(\vx - \vx)^\tp M
     (\vx - \vx) \right) \\
   &\hspace{4em} + \sigma_n^2 \delta_{\vx\vx} \bigg] p(\vx) d\vx \\
 &= \int \left[ \sigma_f^2 \exp({\bf 0}^\tp M {\bf 0}) + \sigma_n^2 \right] p(\vx) d\vx \\
&= \left[ \sigma_f^2 + \sigma_n^2 \right] \int p(\vx) d\vx = \sigma_f^2 + \sigma_n^2
\end{aligned}
\end{equation*}


\subsubsection{Case 3: $\vx\neq\vv$, with \ac{RBF} kernel.}
We substitute into \fref{eq:independent-inputs} the definitions from equations
(\ref{eq:rbf-definition}) and (\ref{eq:pdf-partially-observed}):

\begin{equation*}
\begin{aligned}
 k(\vx, \vv) = \int & \left[\sigma_f^2 \exp\left( -\frac{1}{2}(\vx - \vv)^\tp M (\vx - \vv) \right) + \sigma_n^2 \delta_{\vx\vv} \right] \\
 & \left[\sum_{k=1}^K \tau_k^\vx g_k^\vx \right]
 \left[\sum_{l=1}^L \tau_l^\vv g_l^\vv \right]
 d\vx d\vv
\end{aligned}
\end{equation*}

Rearranging the terms and by linearity of the antidifferentiation:
\begin{equation}
\begin{aligned}
  &k(\vx,\vv) = \sum_{k=1}^K \sum_{l=1}^L \big[\sigma_n^2 \int \delta_{\vx\vv} \; \tau^\vx_k g_k^\vx \; \tau^\vv_l g_l^\vv \; d\vx d\vv \; + \\
  &\hspace{1em}\tau^\vx_k \tau^\vv_l \sigma_f^2 \int \exp\left(-\frac{1}{2}(\vx-\vv)^\tp M (\vx-\vv)\right)
g_k^\vx g_l^\vv d\vx d\vv
\end{aligned}
\label{eq:kxv-2integrals}
\end{equation}

Focusing on the first integral: it is the expectation of a Kronecker delta
$\delta_{\vx\vv}$, with respect to $\vx$ and $\vv$. At least one of $x_i \in \vx$ or
$v_j \in \vv$ follows a Gaussian distribution, the density function of which has
a strictly positive, finite, value on all
the reals. Thus $\E_\vx\E_\vv[\delta_{\vx\vv}] = 0$.

Expanding the second integral, which we name $I_{kl}$:

\begin{equation*}
\begin{aligned}
I_{kl} &= \int \exp\left(-\frac{1}{2}(\vx-\vv)^\tp M (\vx-\vv)\right) g_k^\vx g_l^\vv d\vx
d\vv \\
&= \int \exp\left(-\frac{1}{2}(\vx-\vv)^\tp M (\vx-\vv)\right) \\
&\hspace{2em}\exp\left(-\frac{1}{2}\left(\vx_\text{mis} -
    \mu_k^\vx\right)^\tp(\Sigma_k^\vx)^{-1}\left(\vx_\text{mis} - \mu_k^\vx\right)\right) \\
& \hspace{4em} \delta(\vx_\text{obs}-\mu^\vx_\text{obs}) \\
&\hspace{2em}\exp\left(-\frac{1}{2}\left(\vv_\text{mis} -
    \mu_l^\vv\right)^\tp(\Sigma_l^\vv)^{-1}\left(\vv_\text{mis} - \mu_l^\vv\right)\right)\\
&\hspace{4em} \delta(\vv_\text{obs}-\mu^\vv_\text{obs})
d\vx d\vv \\
&= \int \exp( f_{kl}(\vx, \vv) ) \delta(\vx_\text{obs}-\mu^\vx_\text{obs}) \delta(\vv_\text{obs}-\mu^\vv_\text{obs}) d\vx d\vv
\end{aligned}
\end{equation*}

Where we defined:
\begin{equation*}
\begin{aligned}
  f_{kl}(\vx, \vv) = -\frac{1}{2}\big[ &(\vx-\vv)^\tp M (\vx-\vv) \\
&+ \left(\vx_\text{mis} - \mu_k^\vx\right)^\tp(\Sigma_k^\vx)^{-1}\left(\vx_\text{mis} - \mu_k^\vx\right) \\
&+ \left(\vv_\text{mis} - \mu_l^\vv\right)^\tp(\Sigma_l^\vv)^{-1}\left(\vv_\text{mis} - \mu_l^\vv\right) \big]
\end{aligned}
\end{equation*}

Recall that the integral sign is actually several integrals, and that $d\vx =
\prod_j dx_j$. We can freely permute the integrals, we do so here and separate
observed and missing variables:

\begin{equation*}
  \begin{aligned}
    I_{kl} &= \int \bigg[ \int \exp( f_{kl}([\vx_\text{mis}; \vx_\text{obs}], [\vv_\text{mis}; \vv_\text{obs}]) ) \\
    & \hspace{2em} \delta(\vx_\text{obs}-\mu^\vx_\text{obs}) \delta(\vv_\text{obs}-\mu^\vv_\text{obs}) d\vx_\text{obs} d\vv_\text{obs} \bigg] d\vx_\text{mis}d\vv_\text{mis}
  \end{aligned}
\end{equation*}

Where we made explicit that $\vx = [\vx_\text{mis}; \vx_\text{obs}]$, up to
element permutation.
We can now integrate out the observed variables by replacing them with the
offsets in the Dirac deltas:

\begin{equation*}
  I_{kl} = \int \exp( f_{kl}([\vx_\text{mis}; \mu^\vx_\text{obs}], [\vv_\text{mis}; \mu^\vv_\text{obs}]) ) d\vx_\text{mis}d\vv_\text{mis}
%  \label{eq:f-missing}
\end{equation*}

Now we need to separate $\vx_\mis$ from $\vx_\obs$ in $f$. First we regroup its terms:
\begin{equation*}
\begin{aligned}
f_{kl}(\vx, \vv) &= - \frac{1}{2}\big[(\vx-\vv)^\tp M (\vx-\vv) \\
&\hspace{1.1em}+ (\vx_\text{mis} - \mu_k^\vx)^\tp (\Sigma_k^\vx)^{-1}(\vx_\text{mis} - \mu_k^\vx ) \\
&\hspace{1.1em}+ (\vv_\text{mis} - \mu_l^\vv)^\tp (\Sigma_l^\vv)^{-1}(\vv_\text{mis} - \mu_l^\vv) \big] \\
&= -\frac{1}{2}\big[\vx^\tp M \vx - \vv^\tp M \vx - \vx^\tp M \vv + \vv^\tp M \vv \\
&\hspace{1.1em}+ \vx_\text{mis}^\tp  (\Sigma_k^\vx)^{-1}\vx_\text{mis} - (\mu_k^\vx)^\tp (\Sigma_k^\vx)^{-1} \vx_\text{mis} \\
&\hspace{1.1em}- \vx_\text{mis}^\tp (\Sigma_k^\vx)^{-1} \mu_k^\vx + (\mu_k^\vx)^\tp (\Sigma_k^\vx)^{-1} \mu_k^\vx + v_l \big] \\
&= -\frac{1}{2}\big[\vx^\tp M \vx - \vv^\tp (M+M^\tp) \vx + \vv^\tp M \vv \\
&\hspace{1.1em}+ \vx_\text{mis}^\tp  (\Sigma_k^\vx)^{-1}\vx_\text{mis} + (\mu_k^\vx)^\tp (\Sigma_k^\vx)^{-1} \mu_k^\vx \\
&\hspace{1.1em}- 2 (\mu_k^\vx)^\tp (\Sigma_k^\vx)^{-1} \vx_\text{mis}  + v_l \big] \\
\end{aligned}
\end{equation*}
With a term that depends on $\vv_\mis$ and the cluster $k$, now constant for our purposes:
\begin{equation*}
\begin{aligned}
v_l &= \vv_\text{mis}^\tp  (\Sigma_l^\vv)^{-1}\vv_\text{mis} + (\mu_l^\vv)^\tp (\Sigma_l^\vv)^{-1} \mu_l^\vv \\
&- 2 (\mu_l^\vv)^\tp (\Sigma_l^\vv)^{-1}\vv_\text{mis} \\
\end{aligned}
\end{equation*}

We can split multiplications involving $\vx$:
\begin{equation*}
\begin{aligned}
\vx^\tp M \vx &=
\begin{bmatrix}\vx_\mis \\ \mu^\vx_\obs\end{bmatrix}^\tp
\begin{bmatrix} M_{\misx,\misx} & M_{\misx,\obsx} \\
M_{\obsx,\misx} & M_{\obsx,\obsx}
\end{bmatrix}
\begin{bmatrix}\vx_\mis \\ \mu^\vx_\obs\end{bmatrix} \\
\vx^\tp M \vx &= \vx_\mis^\tp M_{\misx,\misx}\vx_\mis + (\mu^\vx_\obs)^\tp M_{\obsx,\obsx}\mu^\vx_\obs \\
&+ (\mu^\vx_\obs)^\tp \left(M_{\misx,\obsx}^\tp + M_{\obsx,\misx} \right)\vx_\mis  \\
\end{aligned}
\end{equation*}
\begin{equation*}
\begin{aligned}
\vv^\tp(M + M^\tp) \vx &= \vv^\tp(M + M^\tp)_{:,\misx} \vx_\mis \\
&+ \vv^\tp(M + M^\tp)_{:,\obsx} \mu^\vx_\obs
\end{aligned}
\end{equation*}

Substituting these back into $f_{kl}$ we get:
\begin{equation*}
  f_{kl}(\vx, \vv) = -\frac{1}{2}\vx_\mis^\tp \vA \vx_\mis + \va^\tp \vx_\mis + a
\end{equation*}
The newly introduced variables carry an implicit $kl$ index:
\begin{equation*}
\begin{aligned}
\vA &= M_{\misx,\misx} + (\Sigma_k^\vx)^{-1} \\
\va^\tp &= -\frac{1}{2}\Bigg[(\mu^\vx_\obs)^\tp \left(M_{\misx,\obsx}^\tp + M_{\obsx,\misx} \right) \\
&\hspace{1.1em}- 2 (\mu_k^\vx)^\tp (\Sigma_k^\vx)^{-1} - \vv^\tp(M + M^\tp)_{:,\misx} \Bigg] \\
a &= -\frac{1}{2} \Bigg[ (\mu^\vx_\obs)^\tp M_{\obsx,\obsx}\mu^\vx_\obs - (\mu^\vx_\obs)^\tp(M + M^\tp)_{\obsx,:} \vv \\
&\hspace{1.1em} + \vv^\tp M \vv + (\mu_k^\vx)^\tp (\Sigma_k^\vx)^{-1} \mu_k^\vx + v_l \Bigg]
\end{aligned}
\end{equation*}

$I_{kl}$ is a Gaussian integral. These integrals only have a closed form when
integrated over $(-\infty, \infty)$. Fortunately, this is the case here.
For $d$-dimensional vectors $\vx$, $\vb$ and matrix $\vA$
\citep[Equation~11]{straub2009brief}:
\begin{equation}
  \begin{aligned}
\int_{-\infty}^\infty \dots \int_{-\infty}^\infty &\exp\left(-\frac{1}{2}\vx^\tp\vA\vx + \vb^\tp\vx\right) d\vx = \\
&\frac{(2\pi)^{d/2}}{|\vA|^{1/2}} \exp \left(\frac{1}{2} \vb^\tp \vA^{-1}\vb \right)
  \end{aligned}
  \label{eq:gaussian-integral}
\end{equation}

We use this to integrate $\vx_\mis$ in $I_{kl}$, with $f$ as in \fref{eq:gaussian-integral}:
\begin{equation*}
  \begin{aligned}
  I_{kl} &= \int \exp(a) \left[ \int \exp(-\frac{1}{2}\vx_\mis^\tp \vA \vx_\mis + \va^\tp \vx_\mis) d\vx_\mis \right] d\vv_\mis \\
    &= \int \exp(a) \frac{(2\pi)^{|\misx|/2}}{|\vA|^{1/2}} \exp \left(\frac{1}{2} \va^\tp \vA^{-1}\va \right) d\vv_\mis
  \end{aligned}
\end{equation*}

We expand the new multiplication, to separate $\vv$:
\begin{equation*}
\begin{aligned}
  \frac{1}{2} \va^\tp \vA^{-1} \va &= \frac{1}{8}\bigg[
  \vv^\tp(M + M^\tp)_{:,\misx} \vA^{-1} (M + M^\tp)_{:,\misx}^\tp\vv \\
  &- \va_0^\tp \left(\vA^{-1} + (\vA^{-1})^\tp\right) (M + M^\tp)_{:,\misx}^\tp \vv \\
  &+ \va_0^\tp\vA^{-1}\va_0 \bigg] 
\end{aligned}
\end{equation*}
With:
\begin{equation*}
\begin{aligned}
  \va_0^\tp &= (\mu^\vx_\obs)^\tp \left(M_{\misx,\obsx}^\tp + M_{\obsx,\misx} \right) \\
&- 2(\mu_k^\vx)^\tp (\Sigma_k^\vx)^{-1}
\end{aligned}
\end{equation*}

Using this we can rewrite the integral:
\begin{equation*}
\begin{aligned}
I_{kl} = \mathcal{Z}_0 \int \exp \left(\vv^\tp \vB \vv + \vb^\tp \vv + \vv_\mis^\tp \vB_0 \vv_\mis + \vb_0^\tp \vv_\mis \right) d\vv_\mis
\end{aligned}
\end{equation*}
\begin{equation*}
\begin{aligned}
\mathcal{Z}_0 &= \frac{(2\pi)^{|\misx|/2}}{|\vA|^{1/2}}\exp\bigg(-\frac{1}{2}\big[ \\
&\hspace{2em} (\mu_k^\vx)^\tp (\Sigma_k^\vx)^{-1} \mu_k^\vx
+ (\mu_l^\vv)^\tp (\Sigma_l^\vv)^{-1}\mu_l^\vv \\
&\hspace{2em}+ (\mu^\vx_\obs)^\tp M_{\obsx,\obsx}\mu^\vx_\obs \big]
+ \frac{1}{8}\va_0^\tp \vA^{-1} \va_0 \bigg) \\
\vB &= -\frac{1}{2} \left[M - \frac{1}{4}(M + M^\tp)_{:,\misx} \vA^{-1} (M + M^\tp)_{:,\misx}^\tp \right] \\
\vb^\tp &= -\frac{1}{8}\va_0^\tp \left(\vA^{-1} + (\vA^{-1})^\tp\right) (M + M^\tp)_{:,\misx}^\tp \\
&\hspace{2em}+ \frac{1}{2} (\mu^\vx_\obs)^\tp (M + M^\tp)_{\obsx,:} \\
\vB_0 &= -\frac{1}{2} (\Sigma_l^\vv)^{-1} \\
\vb_0^\tp &= (\mu_l^\vv)^\tp (\Sigma_l^\vv)^{-1}
\end{aligned}
\end{equation*}

And rewrite it again, to completely pull out the $\vv_\obs$:
\begin{equation*}
\begin{aligned}
I_{kl} = \mathcal{Z}_1 \int \exp \left(-\frac{1}{2}\vv_\mis^\tp \vC \vv_\mis + \vc^\tp
\vv_\mis \right) d\vv_\mis
\end{aligned}
\end{equation*}
\begin{equation*}
\begin{aligned}
  \mathcal{Z}_1 &= \mathcal{Z}_0 \exp\left((\mu_\obs^\vv)^\tp\vB_{\obsv,\obsv}\mu_\obs^\vv
  + \vb_\obsv^\tp \mu_\obs^\vv \right) \\
\vC &= (\Sigma_l^\vv)^{-1} -2 \vB_{\misv,\misv} \\
\vc &= \vb_0 + \vb_\misv + (\vB + \vB^\tp)_{\misv,\obsv}\mu_\obs^\vv
\end{aligned}
\end{equation*}

We use \fref{eq:gaussian-integral} again:
\begin{equation*}
\begin{aligned}
I_{kl} &= \mathcal{Z}_1 \int \exp \left(-\frac{1}{2}\vv_\mis^\tp \vC \vv_\mis + \vc^\tp
\vv_\mis \right) d\vv_\mis \\
 &= \mathcal{Z}_1 \frac{(2\pi)^{|\misv|/2}}{|\vC|^{1/2}}\exp\left(\frac{1}{2}\vc^\tp \vC^{-1}\vc\right)
\end{aligned}
\end{equation*}

Finally, we substitute this back into \fref{eq:kxv-2integrals}:
\begin{equation}
k(\vx, \vv) = \sum_{k=1}^K\sum_{l=1}^L \sigma_f^2 \tau_k\tau_l I_{kl}
\end{equation}

Before writing an expanded expression for $k(\vx, \vv)$, we simplify some of
its components. The \ac{RBF} distance hyperparameter $M$ is usually a diagonal
matrix, we will only assume it is symmetric. If $M = M^\tp$, then $\vA =
\vA^\tp$, and:
\begin{equation*}
\begin{aligned}
  \vB &= -\frac{1}{2}\left[M - M_{:,\misx}\vA^{-1}M_{\misx,:}\right] \\
  \va_1 &= \frac{1}{2}\va_0 = M_{\misx,\obsx} \mu_\obs^\vx - (\Sigma_k^\vx)^{-1}\mu_k^\vx \\
  \vb &= - M_{:,\misx}\vA^{-1} \va_1 + M_{:,\obsx}\mu^\vx_\obs
\end{aligned}
\end{equation*}

This implies that $\vB$ is symmetric. We also define $\vd$, which will simplify
the constants introduced in $\mathcal{Z}_1$:
\begin{equation*}
\begin{aligned}
  \vC &= (\Sigma_l^\vv)^{-1} + M_{\misv,\misv} - M_{\misv,\misx}\vA^{-1}M_{\misx,\misv} \\
  \vd &= 2\vB_{:,\obsv}\mu_\obsv \\
  \vc &= (\Sigma_l^\vv)^{-1}\mu_l^\vv + \vb_\misv + \vd_\misv
\end{aligned}
\end{equation*}

Recall that $\misx \cap \obsx = \varnothing$, so all the elements of
$M_{\misx,\obsx}$ are off-diagonal in $M$. Thus, if we assume $M$ is diagonal,
as is often the case with the \ac{RBF} kernel, then $M_{\misx,\obsx} = 0$:
\begin{equation*}
\begin{aligned}
  \vC &= (\Sigma_l^\vv)^{-1} + M_{\misv,\misv}\\
  \va_1 &= - (\Sigma_k^\vx)^{-1}\mu_k^\vx \\
\end{aligned}
\end{equation*}


Finally we expand \fref{eq:kxv-2integrals}. Note that $\vA$, $\va_1$,
$\vB$, $\vb$, $\vC$ and $\vc$ depend on $k$ and $l$, and the powers of $(2\pi)$
cancel out.
\begin{equation*}
\begin{aligned}
  k(\vx, \vv) &= \sigma_f^2 \sum_{k=1}^K\sum_{l=1}^L w_k^\vx w_l^\vv
  \left(|\Sigma_k^\vx||\vA||\Sigma_l^\vv||\vC|\right)^{-1/2} \exp \bigg(\\
  &\hspace{1.1em} -\frac{1}{2} \big[
(\mu_k^\vx)^\tp (\Sigma_k^\vx)^{-1} \mu_k^\vx
+ (\mu_l^\vv)^\tp (\Sigma_l^\vv)^{-1}\mu_l^\vv \\
&\hspace{3em}+ (\mu^\vx_\obs)^\tp M_{\obsx,\obsx}\mu^\vx_\obs
- \va_1^\tp \vA^{-1} \va_1  \\
&\hspace{3em}- \vc^\tp\vC^{-1} \vc - (\vd + 2\vb)^\tp_\obsv\mu_\obs^\vv
\big] \bigg) \\
\end{aligned}
\end{equation*}

One final remark. When computing a kernel matrix, it is computationally helpful
to compute the following quantities once per Gaussian, rather than once per pair of Gaussians:
$\vA^{-1}$, which serves as $\vC^{-1}$ when the point is $\vv$ instead of $\vx$,
$(|\Sigma_k^\vx||\vA|)$, $\left[\left(\Sigma_k^\vx\right)^{-1} \mu_k^\vx\right]$,
$\left[\left(\mu_k^\vx\right)^\tp
\left(\Sigma_k^\vx\right)^{-1}\mu_k^\vx\right]$, $\left[\vA^{-1}\va\right]$ and
$\left[\va\vA^{-1}\va\right]$.

\end{document}

%%% Local Variables:
%%% mode: latex
%%% TeX-master: "paper"
%%% End: